\documentclass[12pt]{article}
\pagenumbering{arabic}
\linespread{1.5}
\usepackage[letterpaper, portrait, margin=1in]{geometry}
\usepackage{titling}
\setlength{\droptitle}{-2cm}

\usepackage{fancyvrb}
\usepackage{setspace}
\usepackage{indentfirst}
\usepackage[section]{placeins}
\usepackage[normalem]{ulem}
\useunder{\uline}{\ul}{}
\usepackage{subcaption}
\usepackage{multicol}
\usepackage{float}
\usepackage{multirow}

\usepackage [english]{babel}
\usepackage [autostyle, english = american]{csquotes}
\MakeOuterQuote{"}

\begin{document}

\title{CMPSC 441 Final Project\\An Empirical Study of Distributed Test Suite Execution}
\author{Michael Camara\\Gary Miller\\Herbie Torrance\\Colton McCurdy\\\textit{Honor Code Pledge: This work is ours unless otherwise cited}}
\date{4-11-16}
\maketitle

\begin{abstract}
Testing software is important for ensuring correctness of any system, but it is often the case that test suites take a prohibitive amount of time to complete.  We will analyze a distributed approach for executing test cases, with the goal of achieving improvements in overall response time.  We will develop a novel system for accomplishing this task, in addition to analyzing an existing system called Test Load Balancer.  We will collect and analyze data from both systems so that we can make inferences as to the trade-offs of distributed test suite execution.
\end{abstract}

\section{Overview}
In order to assess the effectiveness of distributed test suite execution, we plan to build a novel system and compare it with an existing system called Test Load Balancer.  Our system will utilize Java RMI and a FileZilla FTP server to allow communication between a client, a registry, and multiple servers, each located on separate nodes.  The node with the registry, hereafter known as the delegator, will host the FTP server and maintain a local directory of resources that will be accessible via this server.  The client will communicate with the delegator via Java RMI and send a list of \texttt{.class} test files, as well as all resources that serve as dependencies for the collection of tests.  The delegator will store the resources in its local filesystem, accessible via the FTP server, and then begin to send the test files to servers for further processing, hereafter known as test servers.  Ideally, the delegator will utilize a load sharing approach to distribute these test files evenly to all connected test servers, waiting until a test server finishes running a test before giving it a new one.  The test servers will each utilize a custom class loader that targets the appropriate directories in the FTP server, and then use JUnit to execute the passed test file.  This will allow the test servers to only access the files they need to run their respective tests from the FTP server, without needing to copy every file to their local filesystems.  The result from running each test will be sent back to the delegator, and then back to the client, thereby completing the distributed test suite execution process.

Work in distributing Java test suites has been attempted by already existing systems, including a tool
named Test Load Balancer (TLB) that we found on GitHub (\textit{https://github.com/\\test-load-balancer/tlb}).
This tool is able to distribute JUnit tests to multiple nodes. These nodes
can be multiple threads on a local machine, multiple virtual machines, or
multiple physical machines. It is capable of evenly distributing tests
to these nodes by looking at the results from previously run tests. It looks
at this data so that it is able to best prune, order and distribute the tests
in future runs. The results from the previously run tests are stored
on the server, which is referenced upon calling the load balancer. This is, of
course, a brief overview and introduction to TLB tool.

In this project we plan to compare the runtime of running test suites locally,
through the TLB tool and finally against Michael's test load sharer. Based upon
these results, we will be able to inform readers of our final project the
best way to run test suites in a distributed fashion.  We will take advantage 
of existing systems with adequate test suites rather
than writing a system to test. There are a number of Java projects on GitHub
with adequate test suites. We will run a number---large enough to gather
adequate results---of these projects' test suites against the aforementioned
systems and compare the runtimes.

A significant portion of our novel system has already been implemented; but if we encounter any significant 
difficulties then we do have a backup plan to pursue. This plan consists of taking advantage of the many features of the TLB
tool, its ability to run a test suite in a distributed fashion on multiple
threads on a local machine, multiple virtual machines, or multiple physical
machines. We would compare the runtimes of these forms of distribution of the tests
to that of running the test suite locally.


\section{Work Distribution}

\begin{itemize}
\item Michael Camara - Michael will implement a system for test suite distribution. Additionally,
Michael will write about this system in our final report.
\item Colton McCurdy - Colton will run the existing TLB tool in the multiple accepted configurations to collect
and analyze data about runtimes. Colton will produce visualizations of these
results using the R programming language.
\item Herbie Torrance - Herbie will run the test suites of the existing projects
on GitHub, locally and analyze those results, as well as results obtained from running our novel system.
\item Gary Miller - Gary will read and write about the TLB tool and any
other tools that he finds will contribute to this project. Gary will also
write about test suite distribution in general.
\end{itemize}

\end{document}