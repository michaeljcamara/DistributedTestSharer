%
% latex-sample.tex
%
% This LaTeX source file provides a template for a typical research paper.
%

%
% Use the standard article template.
%
\documentclass{article}

% The geometry package allows for easy page formatting.
\usepackage{geometry}
\geometry{letterpaper}

% Load up special logo commands.
\usepackage{doc}

% Package for formatting URLs.
\usepackage{url}

% Packages and definitions for graphics files.
\usepackage{graphicx}
\usepackage{epstopdf}
\DeclareGraphicsRule{.tif}{png}{.png}{`convert #1 `dirname #1`/`basename #1 .tif`.png}

%
% Set the title, author, and date.
%
\title{An Analysis of Distributed Test Suites With Load Balancing and Sharing}
\author{Michael Camara, Colton McCurdy, Gary Miller, \& Herbie Torrance}
\date{}

%
% The document proper.
%
\begin{document}

% Add the title section.
\maketitle

% Add an abstract.
\abstract{
This research focuses on the utilizing distributed testing to make the execution of large test suites more efficient. The key trade-offs for taking a distributed approach to testing are be identified and the impacts on using a distributed approach are analyzed. This paper will mainly focus on comparing two distributed test systems, Test Load Balancer (TLB), and our own system, Distributed Test Sharer (DTS).

% Add various lists on new pages.
\pagebreak
\tableofcontents

% Start the paper on a new page.
\pagebreak

%
% Body text.
%
\section{Introduction}
\label{introduction}

Testing is often considered to be one of the most important components of the software development cycle and therefore it is important to conduct a thorough and correct test suite on a software system. Many large software systems are subject to testing that can take several hours and potentially longer to execute, which can be very costly. For example, in situations where a system's test suite can take several hours or even days to execute, and one of the tests that is not executed early in the testing process fails, the cost of identifying this failure can be exceptionally high. For this reason, approaches to decreasing the cost of executing test suites is a very popular topic in software development. One of approaches to successfully reducing the cost associated with testing a system is to use distributed testing. This is an approach in which test cases are distributed or scattered across many different nodes in a network, executed locally on that given node, and the result of the test case is then returned across the network. The focus of this research is to identify, analyze, and report the results of experiments conducted on distributed testing using two different systems, Test Load Balancer (TLB) and Distributed Test Sharer (DTS). It is hypothesized that by distributing test cases and executing them in parallel on several independent nodes, the overall cost of executing a full test suite can be reduced due to the increase of computational resources and evaded the limitations experienced with computing on a single node.

\section{Background On Distributed Testing}

In order to conduct an analysis on using a distributed approach to testing, more research on the trade-offs associated with the topic was conducted. First, it is important to understand situations in which distributed testing may be beneficial; the most obvious being when a test suite takes several hours or days to execute. It also may be necessary to conduct distributed testing if the the system of focus must be tested on different web browsers or sites, in the case of a web-based application, or even on different operating systems. In either case, it may be difficult or costly to executing these tests on a single node and therefore distributed the tests to other nodes may be beneficial. 

Next, it is essential to identify both the benefits and drawbacks of testing in a distributed fashion. Some of the key benefits associated with distributed testing is that less computation is required on the CPU in which the system is executed on, which allows more work to be done locally and is potentially a large benefit in a situation where it is necessary to execute other processes alongside the test suite. More importantly, the goal is to reduce the runtime of the test suite by executing several test cases in parallel and is an exceptionally more feasible task when using multiple CPUs. One of the most common ways in which the increase in performance is a achieved is by diminishing the cost of dependencies. Tests cases that are dependent on another case often have larger costs because their result or even execution may not occur until another case is completed. By using distributed testing, these dependent test cases can be isolated to a particular node that is responsible for executing the group of dependent cases while other nodes can execute other independent cases that otherwise would have been waiting in a queue if centralized testing was used. This situation is a clear example in which a distributed approach would be more beneficial that a centralized approach.

Though it is clear the distributed testing may be beneficial to use there certainly are drawbacks and challenges that exist with this approach. The greatest of these drawbacks is the cost of communication required to transfer data over a network from one node to another. In order to execute a test case at a remote location, either the code to execute the test must exist at the remote location or the code must be sent across the network. This is a potentially costly situation if the required code is large and takes up a lot of memory. The situation where the code already exists at the remote location is often unlikely and therefore the test code usually must be transferred to the node that will execute it, therefore it must be determined whether communication costs of distributed testing out weighs the computational and time costs associated with executing the tests locally. Another drawback associated with distributed testing is that it is dependent on a network, which in some cases may be unavailable. If a system uses a distributed test suite and it is unable to gain access to a network, then there is no way to run the tests until a network becomes available. Another unfavorable situation may occur if there are fewer remote nodes available than required. In a situation such as this, both the communications costs and computational limitations may be experienced because all of the data is transferred to a number of locations that cannot execute the tests in a time that justifies distributing the tests.

Alongside the drawbacks of distributing testing, several challenges exist with this approach as well. One of the key challenges associated with distributed testing is the concept of load sharing versus load balancing. Load sharing occurs when the tests or load are sent to several nodes to be executed without any restrictions on how the load is allocated. Load balancing is a more complex approach in which the allocation of the load is sent to nodes such that each node is doing an equivalent amount of work. This allows the load to be executed more efficiently because it prevents situations in which a particular node is doing a significantly greater amount of work than others or a particular node is doing little to no work at all. It is quite evident that load balancer is a more favorable approach to take, but the complexity associated with it is much larger. For example, consider a situation in which a test case is dependent on another, in this case a load balancing system must not only consider the amount of load to allocate to each node, but also when certain components of the load need to be allocated. 

\section{Overview of Related Work}
\subsection{Existing Systems}

Though distributed testing is a popular and rapidly developing topic in the software industry, not many systems that use his approach exist and work correctly. This is due to the fact that it is not only a relatively new concept, but the complexity of building a system that can run tests on a system and efficiently distribute them to remote nodes is exceptionally high. One existing system that successfully does this is named Test Load Balancer (TLB). TLB claims to support testing of every language on every platform and partitions the tests into subsets that can be executed in parallel on different physical or virtual machines. These subsets are executed in such a way that they all start at the same time and finish at almost the same time as well, therefore the overall time is takes to execute all of the tests can be divided by the number of test subsets that are generated. TLB assures that a particular node will execute only a number of tests proportional to the total number of test divided by the number of available nodes and guarantees mutual exclusion and collective exhaustion of test execution, meaning that no test will be executed more than once and every test will be executed. The TLB system is comprised on two main components, the first of which is a server that is responsible for storing test data and the second is the balancer that partitions and orders the execution of tests. By storing test data, the TLB system will reorder the execution of tests with each run based on previous results. This is a extremely beneficial feature because tests that are known to fail will be executed earlier to avoid unnecessary downtime time resulting from failing tests being executed later. Though it is very important to note that this is only achievable by using tests that are independent of one another which is not always the cases. Unfortunately, large systems very often have test cases that are dependent on other tests and systems such as these are usually examples that would benefit from using distributed testing opposed to centralized testing. Also, this method can not guarantee the order in which tests are executed due to its fail early approach to executing test, therefore tests must not only be independent of other tests but also independent of the order in which they are executed.

\subsection{Distributed Test Sharer}
This section will outline Michael's DTS system and its key features.


\section{Experimentation Protocol}

This section will explain the experimentation process that was followed for collecting data on the two systems.


\section{Analysis of Results}

This section will report and analyze the data that was collected from the experiment.

\section{Challenges}

This section identifies that challenges experienced in completing the project.

\section{Conclusion}

This section concludes the findings and topics outlined throughput the report.

% Generate the bibliography.
\bibliography{latex-sample}
\bibliographystyle{unsrt}

\end{document}
